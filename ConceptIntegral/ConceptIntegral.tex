\documentclass[a4paper,10pt]{report}


\usepackage[T1]{fontenc}
\usepackage{amsmath}
\usepackage{amssymb}
\usepackage{graphicx}
\usepackage{fancybox}

% Title Page
\title{Concept Integral}
\author{Thomas Oswald}


\begin{document}

%\maketitle
%\tableofcontents

%\begin{abstract}



%\end{abstract}
The surface integral for the vector potential:
\begin{equation}
    \mathbf{A}(\mathbf{r)}=\frac{\mu_0}{4 \pi} \int_{S'} \mathbf{I}(\mathbf{r'}) e^{\imath k \mathbf{e}_r(r) \cdot \mathbf{r}'} dS'
\end{equation}
Using the sum over all i wires

\begin{equation}
    \mathbf{A}(\mathbf{r})=\sum_i \mathbf{A}_i(\mathbf{r})
\end{equation}
where
\begin{equation}
    \mathbf{A}_i(\mathbf{r})=\frac{\mu_0}{4 \pi} \int_{r'} \mathbf{I}_i (\mathbf{r'}) e^{\imath k \mathbf{e}_r(\mathbf{r}) \cdot \mathbf{r}'} dr'
\end{equation}

On each wire we know the current at 5 points, which divides the line integral into four parts.

\begin{equation}
    \mathbf{A}(\mathbf{r})=\sum_i \sum_j \mathbf{A}_{ij}(\mathbf{r})
\end{equation}

where $j=1...4$. So

\begin{eqnarray}
    \mathbf{A}(\mathbf{r})&=&\frac{\mu_0 }{4 \pi} \sum_i \sum_j L_{ij}\int_{\rho=0}^{\rho=1} \mathbf{I}_{ij}(\rho) e^{\imath k \mathbf{e}_r(\mathbf{r}) \cdot \mathbf{r}(\rho)} d\rho\\
&=&  \frac{\mu_0 }{4 \pi} \sum_i \mathbf{e}_{L,i} \sum_j L_{ij} \int_{\rho=0}^{\rho=1} I_{ij}(\rho) e^{\imath k \mathbf{e}_r(\mathbf{r}) \cdot \mathbf{r}(\rho)} d\rho \nonumber
\end{eqnarray}

So the integral to be solved is

\begin{equation}\label{integral}
    INT= \int_{\rho=0}^{\rho=1} I_{ij}(\rho) e^{\imath k \mathbf{e}_r(\mathbf{r}) \cdot \mathbf{r}(\rho)} d\rho
\end{equation}

Also

\begin{equation}\label{current_ij}
    I_{ij}(\rho)=I_{ij}+(I_{ij+1}-I_{ij})\rho
\end{equation}



\begin{equation}\label{position}
    \mathbf{r}(\rho)=\mathbf{r}_{ij}+\mathbf{e}_{L,i}L_{ij}\rho
\end{equation}

Substituting (\ref{current_ij}) and (\ref{position}) into (\ref{integral}) we get

\begin{eqnarray}
    INT&=& \int_{\rho=0}^{\rho=1} \left(I_{ij}+(I_{ij+1}-I_{ij})\rho\right) e^{\imath k \mathbf{e}_r(\mathbf{r}) \cdot (\mathbf{r}_{ij}+\mathbf{e}_{L,i}L_{ij}\rho)} d\rho \\
&=&\int_{\rho=0}^{\rho=1} \left(I_{ij}+(I_{ij+1}-I_{ij})\rho\right) e^{\imath k \mathbf{e}_r(\mathbf{r}) \cdot \mathbf{r}_{ij}}e^{\imath k \mathbf{e}_r(\mathbf{r}) \cdot\mathbf{e}_{L,i}L_{ij}\rho} d\rho \nonumber \\
&=&e^{\imath k \mathbf{e}_r(\mathbf{r}) \cdot \mathbf{r}_{ij}}\int_{\rho=0}^{\rho=1} \left(I_{ij}+(I_{ij+1}-I_{ij})\rho\right) e^{\imath k \mathbf{e}_r(\mathbf{r}) \cdot\mathbf{e}_{L,i}L_{ij}\rho} d\rho \nonumber
\end{eqnarray}

Hence there are two integrals to be solved:

\begin{equation}\label{I1}
    I1=\int_{\rho=0}^{\rho=1}I_{ij} e^{\imath k \mathbf{e}_r(\mathbf{r}) \cdot\mathbf{e}_{L,i}L_{ij}\rho} d\rho
\end{equation}
and
\begin{equation}\label{I2}
    I2=\int_{\rho=0}^{\rho=1}\left(I_{ij+1}-I_{ij}\right)\rho e^{\imath k \mathbf{e}_r(\mathbf{r}) \cdot\mathbf{e}_{L,i}L_{ij}\rho} d\rho
\end{equation}

Both are standard integrals. The solution:

\begin{eqnarray}\label{I1_solution}
    I1_{ij}&=&I_{ij}\int_{\rho=0}^{\rho=1}e^{\imath k \mathbf{e}_r(\mathbf{r}) \cdot\mathbf{e}_{L,i}L_{ij}\rho} d\rho \\
&=&\frac{I_{ij}}{\imath k \mathbf{e}_r(\mathbf{r}) \cdot\mathbf{e}_{L,i}L_{ij}} \left[ e^{\imath k \mathbf{e}_r(\mathbf{r}) \cdot \mathbf{e}_{L,i}L_{ij}\rho} \right]_0^1 \nonumber \\
&=&-\frac{\imath I_{ij}}{ k \mathbf{e}_r(\mathbf{r}) \cdot\mathbf{e}_{L,i}L_{ij}}  \left( e^{\imath k \mathbf{e}_r(\mathbf{r}) \cdot \mathbf{e}_{L,i}L_{ij}}-1 \right)  \nonumber
\end{eqnarray}

\begin{eqnarray}\label{I2_solution}
    I2_{ij}&=&\left(I_{ij+1}-I_{ij}\right)\int_{\rho=0}^{\rho=1} \rho e^{\imath k \mathbf{e}_r(\mathbf{r}) \cdot\mathbf{e}_{L,i}L_{ij}\rho} d\rho \\
&=&\frac{\left(I_{ij+1}-I_{ij}\right)}{\left( \imath k \mathbf{e}_r(\mathbf{r}) \cdot\mathbf{e}_{L,i}L_{ij}\right)^2 } \left[ e^{\imath k \mathbf{e}_r(\mathbf{r}) \cdot\mathbf{e}_{L,i}L_{ij}\rho} \left(\imath k \mathbf{e}_r(\mathbf{r}) \cdot\mathbf{e}_{L,i}L_{ij}\rho -1 \right)\right]_0^1 \nonumber \\
&=&-\frac{\left(I_{ij+1}-I_{ij}\right)}{\left( k \mathbf{e}_r(\mathbf{r}) \cdot\mathbf{e}_{L,i}L_{ij}\right)^2 } \left( e^{\imath k \mathbf{e}_r(\mathbf{r}) \cdot\mathbf{e}_{L,i}L_{ij}} \left( \imath k \mathbf{e}_r(\mathbf{r}) \cdot\mathbf{e}_{L,i}L_{ij} -1 \right) +1 \right) \nonumber \\
\end{eqnarray}

As a summary

\begin{equation}
     \mathbf{A}(\mathbf{r})=\frac{\mu_0 }{4 \pi} \sum_i \mathbf{e}_{L,i} \sum_j L_{ij}e^{\imath k \mathbf{e}_r(\mathbf{r}) \cdot \mathbf{r}_{ij}}(I1_{ij}+I2_{ij})
\end{equation}

To avoid the singularities during the computation of edges which point into the direction $\mathbf{e}_r$ one can manipulate the expression as follows.

\begin{eqnarray}
    I1_{ij}&=&-\imath I_{ij}\frac{ e^{\imath k \mathbf{e}_r(\mathbf{r}) \cdot \mathbf{e}_{L,i}L_{ij}}-1}{ k \mathbf{e}_r(\mathbf{r}) \cdot\mathbf{e}_{L,i}L_{ij}}\\
&=&-\imath I_{ij}e^{\imath \frac{1}{2} k \mathbf{e}_r(\mathbf{r}) \cdot \mathbf{e}_{L,i}L_{ij}}\left( \frac{ e^{\frac{1}{2} \imath k \mathbf{e}_r(\mathbf{r}) \cdot \mathbf{e}_{L,i}L_{ij}}-e^{-\frac{1}{2} \imath k \mathbf{e}_r(\mathbf{r}) \cdot \mathbf{e}_{L,i}L_{ij}}}{ k \mathbf{e}_r(\mathbf{r}) \cdot\mathbf{e}_{L,i}L_{ij}} \right)\nonumber \\
&=&-\imath I_{ij}e^{\imath \frac{1}{2} k \mathbf{e}_r(\mathbf{r}) \cdot \mathbf{e}_{L,i}L_{ij}}\left( \frac{ 2 \imath \sin \frac{1}{2} k \mathbf{e}_r(\mathbf{r}) \cdot \mathbf{e}_{L,i}L_{ij}}{ k \mathbf{e}_r(\mathbf{r}) \cdot\mathbf{e}_{L,i}L_{ij}} \right)\nonumber \\
&=& I_{ij}e^{\imath \frac{1}{2} k \mathbf{e}_r(\mathbf{r}) \cdot \mathbf{e}_{L,i}L_{ij}}\left( \frac{  \sin \frac{1}{2}  k \mathbf{e}_r(\mathbf{r}) \cdot \mathbf{e}_{L,i}L_{ij}}{\frac{1}{2} k \mathbf{e}_r(\mathbf{r}) \cdot\mathbf{e}_{L,i}L_{ij}} \right)\nonumber
\end{eqnarray}



\end{document}